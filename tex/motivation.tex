\chapter{Motivation}
\section{Open Spaces}
Seit einigen Jahren werden bei OPITZ-CONSULTING regelmäßig Open Spaces
durchgeführt. Ein Open Space ist eine Veranstaltung, bei der mehrere Personen
zusammen kommen, Themen vorschlagen und diese in einer wie auch immer gearteten
Weise bearbeiten, besprechen oder erlernen. Ein Open Space verteilt sich hierbei
auf mehrere Räume und Timeslots, die zuvor festgelegt werden müssen. Die Themen
werden von den Teilnehmern selbst angeboten und die Teilnehmer entscheiden mit
ihrer Interessenbekundung an einem oder mehreren Themen, wie sich das Programm
gestaltet. (siehe
auch:\footnote{\url{http://en.wikipedia.org/wiki/Open_Space_Technology}}) Falls
die Teilnehmer nach Ablauf der Zeit eines Themas weiteres Interesse anmelden
kann das Thema ad-hoc verlängert und verlegt werden.

Bisher verlief die Planung dieser Open Spaces direkt auf der Veranstaltung
mithilfe von Whiteboards oder Flipcharts. Dieses Verfahren erfordert das sich
die Teilnehmer nach jedem Timeslot am Whiteboard einfinden, um sich über
Änderungen am Zeitplan zu informieren. Dies kostet wertvolle Zeit und
unterbricht den Fluss der Veranstaltung. Das noch größere Problem stellt jedoch
dar, dass diese Änderungen auch von Teilnehmern am Whiteboard gepflegt werden
müssen und so spontane, besonders spannende, Diskussionen, die in der Kaffeküche
beginnen, gar nicht erst eingetragen werden.

Dieses sehr statische Planungsvorgehen wurde dem dynamischen Charakter von Open
Spaces nicht gerecht. Als verteilte Veranstaltung mit ``Realtime''-Aspekt ist
für eine effiziente Organisation ein hohes Maß an Vernetzung unverzichtbar. Um
den Wert der Open Spaces für jeden Teilnehmer zu erhöhen wurde somit das Projekt
FROST begonnen.

%%% Local Variables:
%%% mode: latex
%%% TeX-master: "../ai-projekt"
%%% End:
