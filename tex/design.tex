\chapter{Design}
In diesem Kapitel werde ich auf Design Entscheidungen sowohl im Bezug auf die
Architektur der Anwendung als auch der Nutzeroberfläche eingehen.
\section{Architektur}
\label{sec:architektur}
Als Webanwendung besteht FROST aus einem Backend und einem Frontend. Das Backend
wird dabei auf einem Server ausgeführt der über eine Netzwerkverbindung für die
Clients erreichbar sein muss. Weiterhin wird auf dem Server eine Datenbank
betrieben die für das Backend zur Verfügung steht. Das Frontend wird in den
Browsern der Clients ausgeführt.
\subsection*{Backend}
\label{sec:backend}
Das Backend von FROST wurde in Haskell entwickelt. Haskell ist eine funktionale
Programmiersprache auf die später noch eingegangen wird (TODO: link haskell).
Als Datenbank wird derzeit eine Postgres Datenbank verwendet wobei diese bei
Bedarf leicht durch eine angemessenere Datenbank ausgetauscht werden kann. Das
Backend stellt drei primäre Endpunkte zur Verfügung.

\begin{enumerate}
\item \textbf{Administration:} Bietet Aktionen für die Admin Ansicht an
\item \textbf{Static:} Stellt die statischen Assets des Frontends(HTML, CSS, JS)
  zur Verfügung
\item \textbf{Socket:} Bietet Clients Websocketverbindungen an über die
  Änderungen an den Boards vorgenommen werden können
\end{enumerate}

\subsection*{Frontend}
\label{sec:frontend}
Das Frontend orientiert sich am Pattern des \textit{Unidirectional Dataflows}.

\section{User Interface \& User Experience}
%%% Local Variables:
%%% mode: latex
%%% TeX-master: "../ai-projekt"
%%% End:
