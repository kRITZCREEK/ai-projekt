\chapter{Architektur}
\label{cha:architektur}
In diesem Kapitel werde ich auf Design Entscheidungen im Bezug auf die
Architektur der Anwendung eingehen.

Als Webanwendung besteht FROST aus einem Backend und einem Frontend. Das Backend
wird dabei auf einem Server ausgeführt der über eine Netzwerkverbindung für die
Clients erreichbar sein muss. Weiterhin wird auf dem Server eine Datenbank
betrieben die für das Backend zur Verfügung steht. Das Frontend wird in den
Browsern der Clients ausgeführt.

\section{Haskell Backend}
\label{sec:backend}
Das Backend von FROST wurde in Haskell\footnote{\url{https://www.haskell.org/}}
entwickelt. Haskell ist eine funktionale Programmiersprache auf die später noch
eingegangen wird. Als Datenbank wird derzeit eine Postgres Datenbank verwendet
wobei diese bei Bedarf leicht durch eine andere Datenbank ausgetauscht werden
kann. Das Backend stellt drei primäre Endpunkte zur Verfügung.

\begin{enumerate}
\item \textbf{Administration:} Bietet Aktionen für die Admin Ansicht an
\item \textbf{Static:} Stellt die statischen Assets des Frontends (HTML, CSS, JS)
  zur Verfügung
\item \textbf{Socket:} Bietet Clients Websocketverbindungen an über die
  Änderungen an den Boards vorgenommen werden können
\end{enumerate}

Die Entscheidung Haskell im Backend zu nutzen stützt sich auf die folgenden
Qualitäten und Features von Haskell.

\subsection*{Typsicherheit}
Haskell's Typsystem ist ähnlich fortgeschritten wie das von PureScript und
erlaubt das Enkodieren von wesentlichen Invarianten in der Typsignatur von
Funktionen. Bei der Entwicklung von FROST konnte ich mir dies zu Nutze machen um
strikt die ``reinen'' Funktionen von Funktionen mit Seiteneffekt zu trennen. Rein
bedeutet in diesem Zusammenhang, dass die Funktion ausschließlich Input erhält,
diesen in Output transformiert, ohne die Welt um sich herum zu beeinflussen.

\subsection*{WebSockets}
FROST benötigt WebSockets am Backend, um in Echtzeit Updates an die Clients
verteilen zu können. Das verwendete Haskell Framework
``Yesod''\footnote{\url{http://www.yesodweb.com/}} erlaubt es auf einfache Art
und Weise WebSockets und normale HTTP Endpunkte zu mischen.

\subsection*{Concurrency}
WebSockets müssen im Gegensatz zu klassischen HTTP Requests offen gehalten
werden und erfordern, dass der Server einen Listener für jeden WebSocket
abstellt. Dies ist eines der wichtigsten Kriterien für Haskell. Der
Multithreading Support von Haskell ist anderen Sprachen weit voraus. Werkzeuge
wie Software Transactional
Memory\footnote{\url{https://wiki.haskell.org/Software_transactional_memory}}
werden in FROST ausgiebig genutzt und erlauben es dem Server viele Clients
gleichzeitig und trotzdem performant zu bedienen. Durch die Tools die Haskell
hier zur Verfügung stellt kann FROST auf einem kleinen Server (Single Core,
512MB Ram) laufen und dennoch über 100 Clients gleichzeitig bedienen.

\subsection*{Hohe Ähnlichkeit zu PureScript}
Haskell's Syntax ist der von PureScript sehr ähnlich. Auch semantisch liegen
sich die Sprachen sehr nahe, was es erleichtert zwischen Frontend und Backend
hin und her zu wechseln ohne viel Zeit für den Kontextwechsel zu verlieren.

\subsection*{Guter Support für ``State Machine'' artige Anwendungen}
Da in Haskell alle Variablen/Werte unveränderlich sind, ist es eine übliche
Praxis das Fortschreiten eines Zustandes als eine Folge von Zuständen zu
modellieren, die jedoch alle unabhängig voneinander sind. Haskell's
Datenstrukturen sind darauf ausgelegt diesen Use-Case effizient zu unterstützen.
Eine State Machine lässt sich somit effizient und mit hervorragendem Compiler
Support umsetzen.

Diese Eigenschaft macht Haskell zu einer guten Wahl für Applikationen die sich
Event Sourcing zu Nutze machen.

\clearpage
\section{Event Sourcing}
Event Sourcing ist eine Art den Zustand in einer Applikation zu verwalten und
speichern, bei der man nicht den Zustand selbst speichert, sondern die
Ereignisse die zu diesem Zustand geführt haben. Dieses Verfahren habe ich für
FROST aus den folgenden Gründen gewählt.

\subsection*{Events statt Zustand versenden}
FROST kommuniziert nur aufgetretene Events zwischen Clients und Server. Der
aktuelle Zustand lässt sich herstellen, indem man die bisher aufgetretenen
Events sequentiell abspielt. Von da an werden nur noch neue Events verteilt.
Dies unterstüzt die verteilte Natur von FROST, da bei einer Änderung nicht der
neue Zustand an alle Teilnehmer verschickt werden muss, sondern das größenmäßig
viel kleinere Event.

\subsection*{Möglichkeiten für die Zukunft}
Eventsourcing eröffnet interessante Möglichkeiten. Beispielsweise lässt sich ein
Fehler in der Vergangenheit korrigieren indem man das damalige Event verändert
und die Events neu abspielt. Auch unter einem Reporting Gesichtspunkt ist
Eventsourcing interessant, da die Vergangenheit vollständig aufbewahrt wird und
für Analysen zur Verfügung steht.


\section{PureScript \& JavaScript Frontend}
\label{sec:frontend}
Das Frontend orientiert sich am Pattern des \textit{Unidirectional Dataflows}.
Bei diesem Pattern beschreibt man das Viewlayer als eine reine Funktion, welche
den Applikationszustand auf eine virtuelle grafische Form (Virtual DOM)
abbildet. Diese Virtual DOM wird dann von einem Framework mit der aktuellen
tatsächlichen Ansicht verglichen um die Differenz zwischen den beiden zu
bestimmen. Mithilfe der Differenz wird dann eine minimale Menge von Änderungen
bestimmt, die nötig sind um den aktuellen Zustand in den neuen, noch
virtuellen, zu überführen.

Der aktuelle Applikationszustand wird in einer Statemachine festgehalten. Diese
akzeptiert eine Menge von Aktionen oder Events, und verwendet eine Funktion der
Form \texttt{Aktion -> Applikationszustand -> Applikationszustand} um den
Applikationszustand fortzuschreiten. In FROST sieht diese Methode wie folgt aus:
\begin{figure}
\begin{lstlisting}
evalAction :: Action -> AppState -> AppState
evalAction (AddTopic t) as      = addTopic t as
evalAction (DeleteTopic t) as   = deleteTopic t as
evalAction (AddRoom r) as       = addRoom r as
evalAction (DeleteRoom r) as    = deleteRoom r as
evalAction (AddBlock b) as      = addBlock b as
evalAction (DeleteBlock b) as   = deleteBlock b as
evalAction (AssignTopic s t) as = addTimeslot s t as
evalAction (UnassignTopic t) as =
  let topicslotFilter = (/=) t <<< snd
  in as { timeslots = filterMap topicslotFilter as.timeslots }
evalAction (ReplayActions actions) _ = generateState actions
evalAction (ShowError e) as     = as
evalAction NOP as               = as
\end{lstlisting}
\caption{evalAction}
\end{figure}

Auf der linken Seite des Gleichheitszeichens wird Pattern Matchen gegen die
Actions eingesetzt. Auf diese Weise werden die notwendigen Informationen um den
Zustand fortzuschreiten aus den Actions extrahiert. Beispielsweise enthält die
AddTopic Action ein Topic, dass anschließend von der Methode addTopic in den
Applikationszustand eingefügt wird.

Das Viewlayer bekommt nun die Möglichkeit Actions zu generieren und an die
Statemachine weiterzugeben. Diese generiert einen neuen Applikationszustand und
mithilfe dieses Zustandes wird eine neue View generiert. Weil die View nur durch
Actions auf den Applikationszustand einwirken kann ist der Datenfluss in der
Applikation zyklisch aufgebaut und fließt immer nur in eine Richtung.

%%% Local Variables:
%%% mode: latex
%%% TeX-master: "../ai-projekt"
%%% End:
