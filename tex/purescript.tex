\chapter{PureScript}
Das Frontend von FROST wurde zum Großteil in PureScript entwickelt.\\
Was PureScript als Programmiersprache ausmacht und welche Stärken und Schwächen
PureScript während des Projektes gezeigt hat werde ich in diesem Kapitel
erläutern.
\section{Einführung}
PureScript wird auf der Webseite des Projektes mit folgenden Worten beschrieben:
\begin{quote}
  PureScript is a small strongly typed programming language that compiles to
  Javascript.\footnote{\url{www.purescript.org}}
\end{quote}
Ich würde dieser Beschreibung noch hinzufügen, dass es sich um eine rein
funktionale Programmiersprache handelt und damit einige der guten Eigenschaften
von Haskell übernimmt.
PureScript wirbt mit besonders guter JavaScript Interoperabilität durch das
intuitive Foreign Function Interface.
Beispielsweise lässt sich die JavaScript Funktion \texttt{encodeURI} wie folgt in
PureScript einbinden.

\begin{lstlisting}
foreign import encodeURI :: String $\rightarrow$ String
\end{lstlisting}


\section{Evaluierung}
Im Gegensatz zu Haskell wird PureScript jedoch strict
evaluiert und übernimmt damit die Evaluierungsstrategie von JavaScript. Dies
erlaubt es dem PureScript Compiler lesbares JavaScript zu erzeugen, dass sich
leicht mit dem ursprünglichen Quellcode in Verbingung bringen lässt. Dies macht
sowohl Debugging als auch das Untersuchen der Laufzeit von Algorithmen leichter.

Aber die strikte Evaluierungsstrategie macht PureScript zu einer anderen
Programmiersprache und darauf muss man achten wenn man in PureScript
programmiert. Strictness zwingt einen über die Laufzeitrepräsentation der
verwendeten Datenstrukturen nachzudenken, da beispielsweise unendliche/zyklische
Datenstrukturen ``vollständig'' evaluiert werden und damit den Browser zum
Stillstand bringen.

Die mit undendlichen Datenstrukturen Hand in Hand gehende Rekursion ist in einem
strict Umfeld gefährlich, da sie schnell zu einem Stackoverflow führen kann.
Auch besonders High-Level Konstrukte wie Monadentransformer sind in PureScript
mit Vorsicht zu genießen, da sie in der Regel auf tief geschachtelten Closures
beruhen und den Stack damit genau so belasten wie Rekursion.

Derartige Probleme habe ich in FROST gelöst indem ich für die Datenstrukturen
und die besonders aufwendigen Algorithmen (Rendering) JavaScript Bibliotheken
verwendet habe und PureScript eher die Steuerung als die Berechnungen übernimmt.
Dieses Vorgehen ist typisch für die Verwendung von High-Level-Languages. Die
mächtigen Abstraktionen werden genutzt um die Low-Level Details zu steuern um
damit die Applikationslogik und den mechanischen Akt der Berechnung voneinander
zu trennen.

%%% Local Variables:
%%% mode: latex
%%% TeX-master: "../ai-projekt"
%%% End:
