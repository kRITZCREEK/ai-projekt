\chapter{PureScript}
\label{cha:purescript}
Das Frontend von FROST wurde zum Großteil in PureScript entwickelt. Was
Pure\-Script als Programmiersprache ausmacht und welche Stärken und Schwächen
PureScript während des Projektes gezeigt hat werde ich in diesem Kapitel
erläutern.
\section{Einführung}
PureScript wird auf der Webseite des Projektes mit folgenden Worten beschrieben:
\begin{quote}
  PureScript is a small strongly typed programming language that compiles to
  Javascript.\footnote{\url{www.purescript.org}}
\end{quote}
Ich würde dieser Beschreibung noch hinzufügen, dass es sich um eine rein
funktionale Programmiersprache handelt und damit einige der guten Eigenschaften
von Haskell übernimmt. Im folgenden werde ich auf einige Eigenschaften und
Features von PureScript eingehen.
% PureScript wirbt mit besonders guter JavaScript Interoperabilität durch das
% intuitive Foreign Function Interface.
% Beispielsweise lässt sich die JavaScript Funktion \texttt{encodeURI} wie folgt in
% PureScript einbinden.

% \begin{lstlisting}
% foreign import encodeURI :: String -> String
% \end{lstlisting}

\section{Eigenschaften \& Features}

\subsection*{Unveränderlichkeit}
Variablen sind in PureScript unveränderlich und das folgende Programm wird vom
PureScript Compiler abgewiesen:
\begin{lstlisting}
x = 3
x = 4
\end{lstlisting}
Dies unterscheidet PureScript deutlich von JavaScript, wo dieses Programm gar
kein Problem ist:
\begin{lstlisting}
var x = 3
x = 4
console.log('x ist' + x)
// => x ist 4
\end{lstlisting}

Strikte Unveränderlichkeit von Variablen wie in PureScript hat einige Vor- wie
auch Nachteile die ich im folgenden nennen und anschließend auf FROST beziehen
werde.

\subsubsection*{Vorteile von Unveränderlichkeit}

\begin{itemize}
\item Geringere Komplexität

  Einer der wichtigsten Vorteile von unveränderlichen Variablen ist, dass sie es
  dem Programmierer erlauben über ein Programm nachzudenken ohne es ausführen zu
  müssen. Unveränderliche Variablen verringern die Komplexität in Applikationen,
  indem die Menge der sich bewegenden Teile verringert wird. Außerdem verringern
  unveränderliche Variablen die Abhängigkeiten unter den verschiedenen Modulen
  eines Programmes, da ein einzelnes Module keinen Einfluss auf den Zustand
  eines anderen haben kann.

\item Optimierungen 

  Unveränderliche Variablen erlauben Optimierungen in Hinsicht auf Performance.
  Beispielsweise kann man eine unveränderliche Variable beliebig lang in einem
  Cache speichern, da sie nie ungültig werden wird. Das verwendete Framework für
  das Viewlayer macht sich zu Nutze, dass im Umfeld von unveränderlichen
  Variablen Referenzgleichheit auf Wertgleichheit schließen lässt. Bei einer
  baumartigen Struktur wie der DOM (Document Object Model) lassen sich damit
  ganze Teilbäume als unverändert bestimmen, indem man ihre Elternknoten auf
  Referenzgleichheit überprüft.

\end{itemize}

\subsubsection*{Nachteile von Unveränderlichkeit}

\begin{itemize}
\item Höherer Speicherbedarf

  Da Variablen unveränderlich sind, können ihre Referenzen nicht wieder
  verwendet werden und müssen daher Garbage Collected werden. Dies kann die
  Laufzeit von Programmen verschlechtern und zu Space Leaks führen. Ein Space
  Leak ähnelt dabei einem Memory Leak. Ein Memory Leak tritt auf, wenn ein
  Programm Resourcen allokiert, diese jedoch nicht mehr freigibt und damit immer
  mehr Resourcen ``verschlingt''. Im Gegensatz zum Memory Leak werden allokierte
  Resourcen beim Space Leak zwar wieder freigegeben, jedoch werden sie zu lange
  im Speicher festgehalten und belasten damit Heap und Stack ähnlich wie Memory
  Leaks.
  
\item Rekursion und Stackoverflows

  Im Umfeld von unveränderlichen Variablen ist Rekursion die wichtigste
  Kontrollstruktur. Zu tiefe Rekursion kann jedoch den Stack belasten und zu
  Stackoverflows führen. Es gibt Methoden wie Tail Call Elimination oder
  Trampolining um mit diesen Problemen umzugehen. Diese Methoden erfordern
  jedoch zusätzlichen Aufwand, welchen man mit veränderlichen Variablen nicht
  hätte.
\end{itemize}
\subsubsection*{Fazit}
Das Programmieren mit unveränderlichen Variablen erfordert ein Umdenken des
Programmierers in manchen Bereichen. Die Vorteile überwiegen die Nachteile
jedoch eindeutig, und dies konnte ich auch bei der Entwicklung von FROST immer
wieder feststellen. Der Code ist lesbarer und verständlicher und Debugging ist
nur in wenigen Fällen notwendig. Probleme mit dem erhöhten Speicherbedarf sind
in FROST nicht aufgetretenen, da ich von vorneherein darauf geachtet habe zu
tiefe Rekursionen durch folds oder Tail Call Aufrufe zu vermeiden.

\subsection*{Effekt System}

In PureScript muss eine Funktion die einen Seiteneffekt hat dies in ihrem Typ
widerspiegeln. Dies wird durch den Typ Eff (Extensible Effects) abgebildet.
Beispielsweise hat die Funktion \texttt{log}, die ihr Argument auf die
JavaScript Konsole ausgibt, den Typ \texttt{log :: String -> Eff (console ::
  CONSOLE) Unit}. Dies gibt an, dass die Funktion außer ihrem Rückgabewert
\texttt{Unit} den \texttt{CONSOLE} Effekt verursacht.

\subsection*{Typsystem}
Im Gegensatz zu JavaScript hat PureScript ein statisches Typsystem das bereits
zur Compilezeit viele Fehler und Crashes verhindern kann. PureScript's Typsystem
ist deutlich mächtiger als beispielsweise das von Java und kann dadurch auch als
Design Werkzeug eingesetzt werden.

Weitere Vorteile von PureScript's Typsystem sind, dass die Typen einer
Bibliothek als Dokumentation wirken und weniger Unit Tests notwendig sind um
sich vor Regressionen bei Refactorings zu schützen.
\subsection*{Evaluierungstrategie}
Im Gegensatz zu Haskell wird PureScript strikt evaluiert und übernimmt damit die
Evaluierungsstrategie von JavaScript. Dies erlaubt es dem PureScript Compiler
lesbares JavaScript zu erzeugen, dass sich leicht mit dem ursprünglichen
Quellcode in Verbingung bringen lässt. Dies macht sowohl Debugging als auch das
Untersuchen der Laufzeit von Algorithmen leichter.

Aber die strikte Evaluierungsstrategie macht PureScript zu einer anderen
Programmiersprache und darauf muss man achten wenn man in PureScript
programmiert. Strictness zwingt einen über die Laufzeitrepräsentation der
verwendeten Datenstrukturen nachzudenken, da beispielsweise unendliche/zyklische
Datenstrukturen ``vollständig'' evaluiert werden und damit den Browser zum
Stillstand bringen.

Die mit undendlichen Datenstrukturen Hand in Hand gehende Rekursion ist in einem
strict Umfeld gefährlich, da sie schnell zu einem Stackoverflow führen kann.
Auch besonders High-Level Konstrukte wie Monadentransformer sind in PureScript
mit Vorsicht zu genießen, da sie in der Regel auf tief geschachtelten Closures
beruhen und den Stack damit genau so belasten wie Rekursion.

Derartige Probleme habe ich in FROST gelöst indem ich für die Datenstrukturen
und die besonders aufwendigen Algorithmen (Rendering) JavaScript Bibliotheken
verwendet habe und PureScript eher die Steuerung als die Berechnungen übernimmt.
Dieses Vorgehen ist typisch für die Verwendung von High-Level-Languages. Die
mächtigen Abstraktionen werden genutzt um die Low-Level Details zu steuern um
damit die Applikationslogik und den mechanischen Akt der Berechnung voneinander
zu trennen.

%%% Local Variables:
%%% mode: latex
%%% TeX-master: "../ai-projekt"
%%% End:
